\documentclass{amsart}

\usepackage{amssymb}
\usepackage{multicol}
\usepackage{lipsum}% dummy text
%% \usepackage[a4paper,margin=0.75in,landscape]{geometry}
\usepackage[a4paper,margin=0.5in,landscape]{geometry}
\hoffset=-0.2in
%% \voffset=-0.2in

%% \nopagenumbers

\setlength{\columnseprule}{0.4pt}

\newcommand{\Section}[1]{\filbreak\par\vspace{5pt}\noindent\textbf{#1}\par\noindent\ignorespaces}
\newcommand{\us}{\char`\_}  %% underscore
\newcommand\textnit[1]{\textnormal{\textit{#1}}}

\def\ver{3.0}
\def\sagever{8.1}

\begin{document}

\begin{multicols}{3}
\Section{\LARGE{Sage Dynamics Ref Card v\ver}\\
\normalsize{(for Sage \sagever})}


\Section{Rings and Fields}
\noindent
\begin{tabular}{@{\ttfamily\enspace}llll}
ZZ & integer ring & \texttt{Zmod}($m$) & $\mathbb{Z}/m\mathbb{Z}$ \\
QQ & rational field  & \texttt{QQbar} & alg. clos. of QQ \\
RR & real field & \texttt{CC} & complex field \\
Qp($p$) & $p$-adic field & \texttt{Zp($p$)} & $p$-adic integers\\
QQ[] & polynomials & \texttt{QQ[[]]} & power series\\
GF($p$) & prime field & \texttt{GF($p^n$,'\textit{v}')}& finite field\\
\end{tabular}
\begin{tabular}{@{\ttfamily\enspace}ll}
CyclotomicField(\textit{n}) & $\mathbb{Q}(\zeta_n)$\\
FractionField(\textit{ring}) & field of fractions\\
QuadraticField(d) & $\mathbb{Q}(\sqrt{d})$\\
NumberField(\textit{poly},'\textit{var}',[\textit{emb}]) & number field\\
K.absolute\us field() & ---\\
K.degree() & $[K : \mathbb{Q}]$\\
K.extension(poly) & fld ext\\
QQ.range\us by\us height(bd) & iterator\\
\multicolumn{2}{l}{\texttt{K.elements\us of\us bounded\us height(bd, [params])}}\\
\multicolumn{2}{l}{\texttt{number\us field\us elements\us from\us algebraics(\textit{pts})}}
\end{tabular}

\Section{Spaces and Schemes}
\noindent
\begin{tabular}{@{\ttfamily\enspace}ll}
A.<\textit{vars}>=AffineSpace(\textit{ring},\textit{dim}) & $\mathbb{A}^n$\\
P.<\textit{vars}>=ProjectiveSpace(\textit{ring},\textit{dim}) & $\mathbb{P}^n$\\
\multicolumn{2}{l}{\texttt{PP.<\textit{vars}>=ProductProjectiveSpaces(\textit{ring},\textit{dims})}}\\ \multicolumn{2}{r}{$\mathbb{P}^n \times \cdots \times \mathbb{P}^m$}\\
WehlerK3Surface(\textit{polys}) & ---
\end{tabular}
\noindent
\begin{tabular}{@{\ttfamily\enspace}ll}
S.affine\us patch($i$,[$\mathbb{A}$]) & ---\\
S.base\us ring() & base ring S\\
S.change\us ring() & change base ring\\
S.coordinate\us ring() & coor. ring of S\\
S.defining\us ideal() & ---\\
S.defining\us polynomials() & ---\\
S.dimension() & rel. dim of S\\
S.gens() & vars of coord. ring\\
\multicolumn{2}{l}{\texttt{S.point\us transformation\us matrix([pts,pts])}} \\ \multicolumn{2}{r}{find PGL element}\\
S.projective\us embedding([$i$,$\mathbb{P}$]) & ---\\
S.projective\us closure([$i$,$\mathbb{P}$])& --- \\
S.rational\us points([bd,fld]) & ---\\
S.subscheme(\textit{polys}) & subscheme of S\\
S.vars() & vars of coord. ring\\
S.variable\us names() & vars as strings\\
S.weil\us restriction() & restric. of const.
\end{tabular}

\Section{Dynamical System Initialization}
\noindent
\begin{tabular}{@{\ttfamily\enspace}ll}
\multicolumn{2}{l}{\texttt{DynamicalSystem(polys, [domain])}}\\
\multicolumn{2}{r}{projective if no domain}\\
\multicolumn{2}{l}{\texttt{DynamicalSystem\_affine(polys, [domain])}}\\
\multicolumn{2}{l}{\texttt{DynamicalSystem\_projective(polys, [domain])}}\\
f.as\_dynamical\_system() & End $\to$ DS
\end{tabular}

\Section{Periodic Behavior}
\noindent
\begin{tabular}{@{\ttfamily\enspace}ll}
f.dynatomic\us polynomial([$m$,$n$]) & ---\\
Q.is\us preperiodic($f$) & ---\\
Q.multiplier(f,n) & $(f^n)'(Q)$\\
Q.orbit\us structure($f$) & [tail,period] \\
\end{tabular}
\par\noindent
\begin{tabular}{@{\ttfamily\enspace}ll}
f.periodic\us points($n$,[params]) & ---\\
f.rational\us periodic\us points([params]) & ---\\
f.rational\us periodic\us graph([params]) & ---\\
f.rational\us preperiodic\us points([params]) & ---\\
f.rational\us preperiodic\us graph([params]) & ---\\
\multicolumn{2}{l}{\texttt{f.possible\us periods([params])} \quad via good red.}
\end{tabular}

%% \columnbreak

\Section{Heights and Measures}
\noindent
\begin{tabular}{@{\ttfamily\enspace}ll}
Q.canonical\us height($f$,[params]) & $\hat{h}_f(Q)$\\
f.critical\us height() & $\sum_{c \in \text{Crit}} \hat{h}_f(c)$\\
Q.global\us height([prec]) & $h(Q)$\\
f.global\us height([prec]) &  ---\\
Q.green\us function($v$,[prec]) &  at $v$\\
\multicolumn{2}{l}{\texttt{f.height\us difference\us bound()} \quad $\bigl|h(Q) - \hat{h}_f(Q)\bigr|$}\\
f.local\us height\us arch($i$,[prec]) & at $\infty$
\end{tabular}

\Section{Critical Points}
\noindent
\begin{tabular}{@{\ttfamily\enspace}ll}
f.critical\us points() & ---\\
f.critical\us subscheme() & ---\\
f.critical\us point\us portrait() & ---\\
f.critical\us height() & $\sum_{c \in \text{Crit}} \hat{h}_f(c)$\\
f.is\us postcritically\us finite() & ---\\
f.wronskian\us ideal() & crit locus
\end{tabular}

\Section{Cyclic Structures}
\begin{tabular}{@{\ttfamily\enspace}ll}
f.all\us rational\us preimages(\textit{points}) & ---\\
f.cyclegraph() \textnormal{\tiny{\framebox{$\mathbb{F}_q$}}}& digraph \\
Q.orbit\us structure($f$) \textnormal{\tiny{\framebox{$\mathbb{F}_q$}}}& [tail,per]\\
Q.rational\us preimages($f$) & --- \\
Q.rational\us connected\us component(f) & ---
\end{tabular}

\Section{Rational Functions}
\begin{tabular}{@{\ttfamily\enspace}ll}
f.dynamical\_degree() & ---\\
f.degree\_sequence() & deg. of iterates\\
f.indeterminacy\_locus() &  ---\\
f.indeterminancy\us points() & if fin. many
\end{tabular}


\Section{Functions}
\begin{tabular}{@{\ttfamily\enspace}ll}
f[i] & $i$th coord\\
f.automorphism\us group() & $\{\phi : f^{\phi} = f\}$\\
f.autmorphism\us group() & $\text{Hom}(f,f)$\\
f.base\us ring() & ---\\
f.change\us ring() & ---\\
P.chebyshev\us polynomial($k$,\textit{kind}) & \\
f.codomain() & ---\\
f.conjugate($\phi$) & $\phi^{-1} \circ f \circ \phi$\\
f.conjugating\us set($g$) & $\text{Hom}(f,g)$\\
f.defining\us polynomials() & ---\\
f.degree() & ---\\
f.dehomogenize($k$) & --- \\
f.domain() & ---\\
f.homogenize($k$) \\
f.is\us morphism()  & ---\\
f.normalize\us coordinates()& remove gcd\\
f.nth\us iterate($Q$,$n$) & $f^n(Q)$\\
f.nth\us iterate\us map(n) & $f^n$\\
P.Lattes(E,m) & create Latt\`es map\\
f.orbit($Q$,[$m$,$n$]) & $\{f^m(Q),\ldots, f^n(Q)\}$\\
f.primes\us of\us bad\us reduction() & ---\\
f.resultant() & ---\\
f.scale\us by($t$) & $t\cdot f$\\
f.specialization() & subs value of param
\end{tabular}

\Section{Points}
\begin{tabular}{@{\ttfamily\enspace}ll}
Q[i] & $i$th coord\\
Q.change\us ring() & ---\\
Q.clear\us denominator()  & ---\\
Q.codomain() & ambient space\\
Q.dehomogenize($i$) & --- \\
Q.domain() & base ring\\
Q.homogenize($i$) & --- \\
Q.normalize\us coordinates() & remove gcd\\
Q.nth\us iterate($f$,$n$) & $f^n(Q)$\\
Q.orbit($f$,($m$,$n$)) & $\bigl[f^m(Q),\ldots, f^n(Q)\bigr]$\\
Q.scale\us by($t$) & $t\cdot Q$
\end{tabular}

\Section{Iteration}
\begin{tabular}{@{\ttfamily\enspace}ll}
f.nth\us iterate($Q$,$n$) & $f^n(Q)$\\
f.nth\us iterate\us map(n) & $f^n$\\
f.orbit($Q$,[$m$,$n$]) & $\bigl[f^m(Q),\ldots, f^n(Q)\bigr]$\\
f.rational\us preimages($Q$,$k$) & $f^{-k}(Q)$
\end{tabular}

\Section{Moduli Spaces}
\begin{tabular}{@{\ttfamily\enspace}ll}
f.is\us polynomial() & has tot. ram. fixed pt.\\
f.is\us PGL\us minimal() & \\
f.is\us conjugate($g$) & \smash[t]{$g \stackrel{?}{=}f^{\phi}$}\\
f.normal\us form() & $x^n + a_{n-2}x^{n-2} + \cdots + a_0$\\
f.minimal\us model() & min resultant $f^{\phi}$
\end{tabular}
\noindent
\begin{tabular}{@{\ttfamily\enspace}ll}
f.multiplier\us spectra($n$,[params]) \\
  \hfill $\{\lambda_f(Q) : Q \in \text{Per}_n\}$ \\
f.sigma\us invariants($n$,[params]) \\
   \hfill$\{\sigma_i(\lambda_f(Q)) : Q \in \text{Per}_n\}$
\end{tabular}


\Section{Finite Fields}
\begin{tabular}{@{\ttfamily\enspace}ll}
f.cyclegraph() & iteration digraph\\
Q.orbit\us structure(f) & [tail,period]
\end{tabular}

\Section{Mandelbrot and Julia Sets}
\begin{tabular}{@{\ttfamily\enspace}ll}
external\_ray(v) & list or single angle\\
mandelbrot\_plot([params]) & for $z^2+c$\\
julia\_plot([params]) & for $z^2+c$\\
\end{tabular}



\Section{Miscellaneous / Help}
\noindent
\begin{tabular}{@{\ttfamily\enspace}ll@{\ttfamily\enspace}ll}
\us  & last output\\
\%time & execution time\\
timeit('\textit{cmd}',number=\#) & time multiple iterations\\
s.<$\textit{tab}$> & show all cmds on $s$\\
s.\textit{cmd}? & info about cmd on $s$ \\
set\us verbose(None) & disable warnings\\
load(``\textit{path to file}'') & load code file\\
copy(\text{obj}) & ---\\
latex(\text{obj}) & ---\\
all(\textit{list of bool}) & ---\\
any(\textit{list of bool}) & ---\\
sum(\textit{list}) & ---\\
max(\textit{list}) & ---\\
isinstance(f, \textit{type}) & check for type\\
preparser(\textit{bool}) & on/off notebk preparsing
\end{tabular}

\Section{Matrices}
\noindent
\begin{tabular}{@{\ttfamily\enspace}ll@{\ttfamily\enspace}ll}
matrix(K,n,m,\textit{list}) & create matrix\\
matrix(K,\textit{list of lists}) & create matrix\\
M.charpoly() & ---\\
M.determinant() & ---\\
M.height() & global height\\
M.inverse() & ---\\
M.LLL([args]) & LLL reduced lattice\\
M.minors(k) & dets of $k\times k$ minors\\
M.rank() & ---
\end{tabular}

\Section{Polynomial Rings}
\noindent
\begin{tabular}{@{\ttfamily\enspace}ll@{\ttfamily\enspace}ll}
R.<a,b>=PolynomialRing(K,2) & poly ring over K\\
R.<a>=PolynomialRing(K) & univar poly ring\\
R.<a>=PolynomialRing(K,1) & multivar poly ring\\
R.gen($k$) & kth variable\\
R.gens() & all variables\\
R.hom(\textit{im\us gens},S) & Hom(R,S)\\
R.ideal(polys) & ---\\
I.dimension() & krull dim of R/I\\
I.elimination\us ideal(\textit{vars}) & ---\\
I.gens() & ---\\
I.groebner\us basis() & ---\\
I.is\us prime() & ---\\
I.is\us maximal() & ---\\
I.is\us principal() & ---\\
I.is\us one() & ---\\
I.primary\us decomposition() & ---\\
I.radical() & ---\\
I.ring() & R\\
I.variety() & rat pts of dim 0 \\
I.vector\us space\us dimension() & of R/I\\
F.monomial\us coefficient(\textit{mon}) & base ring element\\
F.polynomial(x) & make univariate\\
F.subs(\textit{dict}) & substitution\\
F(\textit{tuple}) & substitution\\
\end{tabular}
\begin{tabular}{@{\ttfamily\enspace}ll@{\ttfamily\enspace}ll}
F.coefficient(\textit{mon}) & poly element\\
F.coefficients() & ---\\
list(F) & list of (coeff,mon)\\
F[\textit{list}] & coeff of mon with exp \textit{list}\\
F.dict() & dict of mon:coef via exp\\
F.lift(I) & coeff of gens of I to get F
\end{tabular}

\Section{Algebraic Geometry}
\noindent
\begin{tabular}{@{\ttfamily\enspace}ll@{\ttfamily\enspace}ll}
S.Chow\us form() & associated Chow form\\
S.coordinate\us ring() & ---\\
S.defining\us ideal() & ---\\
S.defining\us polynomials() & ---\\
S.degree() & from lc of hil poly\\
S.dimension() & relative dimension\\
S.intersection(T) & ---\\
S.intersection\us multiplicity(T,Q) & Serre's Tor\\
S.irreducible\us components() & ---\\
S.is\us smooth() & ---\\
S.Jacobian() & Jacobian ideal\\
S.projective\us closure([$\mathbb{P}$]) & --- \\
S.rational\us points([bd]) & ---\\
S.subscheme(\textit{ideal}) & --- \\
S.veronese\_embedding(d) & ---\\
S.weil\us restriction() & ---\\
S*T & $S \times T$\\
S**n & $S \times \cdots \times S$\\
I.radical() & radical ideal\\
%% \end{tabular}
%% \begin{tabular}{@{\ttfamily\enspace}ll@{\ttfamily\enspace}ll}
\multicolumn{2}{l}{------------------------------} \\
PP.components() & ---\\
PP.dimension\us components() & list of dims\\
PP.segre\us embedding([\textit{codomain}]) & ---\\
%% \end{tabular}
%% \begin{tabular}{@{\ttfamily\enspace}ll@{\ttfamily\enspace}ll}
\multicolumn{2}{l}{------------------------------} \\
C.arithmetic\us genus() & ---\\
C.genus() & ---\\
C.is\us complete\us intersection() & ---\\
C.is\us ordinary\us singularity(Q) & ---\\
C.is\us transverse(D,Q) & ---\\
C.tangents(Q) & 
\end{tabular}


\vspace{5pt}
\noindent
\footnotesize{Copyright \copyright \; 2017 B. Hutz, J. Silverman
  v\ver. Permission is granted for noncommercial distribution provided
  the copyright notice and this permission notice are preserved on all
  copies. Thanks to ICERM for hosting us while version 1.0 was written.}
\end{multicols}
\newpage

\end{document}
